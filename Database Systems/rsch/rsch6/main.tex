\documentclass[12pt, a4paper]{article}
\usepackage{minted}
\usepackage{multirow}
\usepackage{enumerate}
\usepackage{geometry}
%\geometry{left=5cm,right=5cm,top=2.5cm,bottom=2.5cm}
\usepackage{fontspec}
\setmainfont{Times New Roman}
\usepackage{minted}
\usepackage[slantfont,boldfont]{xeCJK}
\setCJKmainfont{SimSun}
%% \usepackage{indentfirst}
%% \setlength{\parindent}{2em}
\usepackage{float}
\usepackage{titling}
\usepackage{graphicx}
\usepackage{subfigure}
\usepackage[square,sort,comma,numbers]{natbib}
\usepackage{booktabs}
\usepackage{amsmath}
\usepackage{titlesec}
\usepackage{url}
\setCJKmonofont{SimHei}
\input zhwinfonts
\setlength{\bibsep}{0.5ex}

\titleformat{\section}{\normalsize\bfseries}{}{0em}{}
\titlespacing{\section}{0pt}{1em}{0em}

\pretitle{\begin{center}\LARGE}
\posttitle{\par\end{center}\vskip 0.5em}
\preauthor{\vspace{8cm}\begin{center}
    \large \lineskip0.5em %
    \begin{tabular}[t]{c}}
\postauthor{\end{tabular}\par\end{center}}
\predate{\begin{center}\large}
\postdate{\par\end{center}}

\begin{document}

\title{{\bf\Huge Analysis of Different Implementations of Bitmap Index}}
\author{\emph{Harbin Institute of Technology}\\\emph{Department of Computer Science and Technology}\\Ma Yukun\\1150310618}

\date{2017/11/18}

\nocite{*}

%% \maketitle\setcounter{page}{0}\thispagestyle{empty}
%% \newpage
%% \tableofcontents

\begin{center}
{\textbf{
Yukun Ma\\
Week \#7: NoSQL Data Stores\\
Page \#1}}
\end{center}

\section{Problem Statement}

NoSQL data stores are very popular these days. This article will show why they are so popular. This article will also discuss on the problems that nosql data stores can solve while the relational databases cannot.

\bibliographystyle{unsrt}
\bibliography{ref}

\section{Overview of NoSQL Data Stores}

NoSQL data stores are data stores that is modeled not in tabular relations which are used in relational databases.\cite{wiki} If they are not modeled in tabular relations, then what are the structure of data stored in a NoSQL data store?

There are several types of data models used by nosql data stores. For example, key-value model is one of the simplest model used in NoSQL data stores.\cite{wiki} And NoSQL data stores that use key-value models are called key-value stores. Similarly, there are document stores, graph databases, object databases and so on.

\section{Critical Thinking}

Let's compare NoSQL data stores with relational databases on the problems that NoSQL data stores can solve while relational databases cannot.

First, NoSQL data stores can free people from design predefined schema. While users of relational databases must define the schema first before add any information to the database.\cite{diff}

Second, NoSQL data stores can deal with the unstructured data such as videos, audios and email which makes up much of data on the Internet todays. While, relational databases can only handle structured data.\cite{diff}

Finally, NoSQL data stores is much easier to ``horizontal'' scaling to clusters of servers than relational databases due to the simpler design of the NoSQL data stores.\cite{wiki}


\newpage

\begin{center}
{\textbf{
Page \#2}}
\end{center}

\section{Question}



\section{Method}

{\emph{Describe how you are going to answer your own question stated above.}}

\section{Analysis and Discussion}

Among the three kinds of data models, the conceptual data models are the most high-level one. It hides the detailed information of database that users want. They usually include the important entities and the relationships among them. But in a conceptual data model, neither attribute nor primary key is specified.

While for a logical data model, things are getting more detailed. A logical data model will describe the data in the database as exhaustively as possible but do not care how to implement. It contains all entities and relationships among the entities. Compared with the conceptual data model, a logical data model will specify all the attributes primary keys, foreign keys, and so on.

A physical data model will not only describe the data in the database in detail, but also will show how the database will be built. Apart from all the information that a logical data model provides, a physical data model will specify all table names, column names, and column types. It is the most detailed among the three kinds data models above.

When we only need to have a general understanding of the concept and structure of the database, we can use the conceptual data model. A conceptual data model is usually the draft of the design of database. Usually we need a more detailed design so that we can evaluate and analyze the rationality and correctness of our design. That's we we should use a logical data model. And before we impelment our design, we should get a physical data model.


%% \renewcommand\refname{Reference}
\end{document}
