\documentclass[12pt, a4paper]{article}
\usepackage{minted}
\usepackage{multirow}
\usepackage{enumerate}
\usepackage{geometry}
%\geometry{left=5cm,right=5cm,top=2.5cm,bottom=2.5cm}
\usepackage{fontspec}
\setmainfont{Times New Roman}
\usepackage{minted}
\usepackage[slantfont,boldfont]{xeCJK}
\setCJKmainfont{SimSun}
%% \usepackage{indentfirst}
%% \setlength{\parindent}{2em}
\usepackage{float}
\usepackage{titling}
\usepackage{graphicx}
\usepackage{subfigure}
\usepackage[square,sort,comma,numbers]{natbib}
\usepackage{booktabs}
\usepackage{amsmath}
\usepackage{titlesec}
\usepackage{url}
\setCJKmonofont{SimHei}
\input zhwinfonts
\setlength{\bibsep}{0.5ex}

\titleformat{\section}{\normalsize\bfseries}{}{0em}{}
\titlespacing{\section}{0pt}{1em}{0em}

\pretitle{\begin{center}\LARGE}
\posttitle{\par\end{center}\vskip 0.5em}
\preauthor{\vspace{8cm}\begin{center}
    \large \lineskip0.5em %
    \begin{tabular}[t]{c}}
\postauthor{\end{tabular}\par\end{center}}
\predate{\begin{center}\large}
\postdate{\par\end{center}}

\begin{document}

\title{{\bf\Huge Analysis of Different Implementations of Bitmap Index}}
\author{\emph{Harbin Institute of Technology}\\\emph{Department of Computer Science and Technology}\\Ma Yukun\\1150310618}

\date{2017/11/18}

\nocite{*}

%% \maketitle\setcounter{page}{0}\thispagestyle{empty}
%% \newpage
%% \tableofcontents

\begin{center}
{\textbf{
Yukun Ma\\
Week \#6: Data Modeling\\
Page \#1}}
\end{center}

\section{Problem Statement}



\bibliographystyle{unsrt}
\bibliography{ref}

\section{Overview of Data Modeling}


\section{Critical Thinking}

%%\footnotetext[3]{data in OLAP systems is from OLTP systems}
\newcommand{\tabincell}[2]{\begin{tabular}{@{}#1@{}}#2\end{tabular}}
\begin{table}[H]
\centering
\caption{The Differences Between IE, IDEF1X, UML, and E-R Diagram}\label{tab:diff}
\begin{tabular}{lccc}
  \toprule
  \emph{Method} &\emph{Entities} &\emph{Attributes} &\emph{Relationships}\\
  \midrule
  IE &square-cornered rectangles &null &solid lines; no FKs\\
  IDEF1X &\tabincell{c}{sqaure-cornered or\\ round-cornered rectangles} &null &solid or dashed lines\\ %%\footnotemark[3]\\
  UML &square-cornered rectangles &inside entity boxes &solid lines\\
  E-R Diagram &squared-cornered rectangle &\tabincell{c}{in circles outside\\ entity boxes} & rhombus symbol\\
  \bottomrule
\end{tabular}
\end{table}

\newpage

\begin{center}
{\textbf{
Page \#2}}
\end{center}

\section{Question}

When I looked up for the information of data modeling, I found there are 3 types of data models, namely conceptual data models, logical data models, and physical data models.\cite{wiki}.

What are these three kinds of data models and under what circumstances should we use each of them?

\section{Method}

{\emph{Describe how you are going to answer your own question stated above.}}

\section{Analysis and Discussion}



%% \renewcommand\refname{Reference}
\end{document}
